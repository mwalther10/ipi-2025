\textbf{Gruppenabgaben:}

\begin{itemize}
    \item Die Aufgaben sollten in Gruppen von idealerweise drei Studierenden bearbeitet und abgegeben werden. Nutzen Sie in Moodle das Forum \glqq ZettelpartnerInnen-Börse\grqq{}, um weitere Mitglieder für eine Gruppe zu finden oder eine Gruppe zu bilden.

    \item Die Mitglieder einer Gruppe sollen von dem ersten Übungsblatt an für die weiteren Übungszettel fix bleiben. 

    \item Ab dem zweiten Übungszettel werden wir Gruppen in Moodle einrichten. Für den ersten Übungszettel reicht es, wenn ein Mitglied einer Gruppe die Lösungen auf Moodle hochlädt. Reichen Sie Ihre Abgabe ausschließlich über die bereitgestellte Upload-Funktion in Moodle ein. 

    \item Damit wir bei jedem Übungsblatt wissen, wer die Gruppenmitglieder sind, ist jeder Abgabe eine einfache Text-Datei mit dem Namen \texttt{mitglieder.txt} beizufügen, in der Nachname, Vorname, Matrikelnummer und Nummer der Übungsgruppe eines jeden Gruppenmitglieds aufgeführt sind. Diese Datei ist mit Teil der Abgabe!
\end{itemize}


\textbf{Für die Abgabe ihrer Lösungen beachten Sie bitte folgende Regeln:}

\begin{enumerate}
    \item Verwenden Sie nur Befehle, Funktionen und Programmiertechniken, die in den bisherigen Vorlesungen (bis zum Abgabetermin) und bisherigen Übungsblättern behandelt wurden.

    \item Die einzelnen Aufgaben sind mit vollständigen Dateinamen versehen. Verwenden Sie genau diese Namen und die dazugehörigen Dateiformate für Ihre Abgabe. Um Tippfehler zu vermeiden, können Sie die Dateinamen auch einfach kopieren.

    \item Ihre Dateien sollen immer im Plaintext-Format (UTF-8 codiert) vorliegen. Python-Code speichern Sie in Dateien mit der Endung \texttt{.py}. Für Texte können Sie zwischen \texttt{.txt} und \texttt{.md} (Markdown) frei wählen. Markdown-Dateien bieten gegenüber einfachen \texttt{.txt}-Dateien zusätzliche Formatierungsmöglichkeiten, die Sie in Visual Studio Code mit der Tastenkombination STRG + SHIFT + V (unter Windows) bzw.~COMMAND + SHIFT + V (unter MacOS) anzeigen lassen können. \textbf{Insbesondere sind also keine PDFs, keine Word-Dokumente und auch keine Bildschirmfotos erlaubt!}

    \item Geben Sie Ihre Aufgaben bis zur angegebenen Deadline über die Moodle-Seite zur Vorlesung ab. Abgaben per Mail werden generell nicht berücksichtigt.

    \item Für Python-Code prüfen wir, ob Ihr Code die Stilrichtlinien von flake8\footnote{Details dazu finden Sie hier: \url{https://flake8.pycqa.org/en/latest/}} befolgt. Dieser Check muss erfolgreich sein; ansonsten gibt es Punkteabzug.

    Eine einfache Möglichkeit, diese Anforderungen zu erfüllen, besteht darin, alle Dateien mit Visual Studio Code zu erstellen oder zu bearbeiten. Wenn Sie zum Beispiel eine Datei mit der Endung \texttt{.txt} oder \texttt{.md} bearbeiten, behandelt Visual Studio Code diese wie eine normale Textdatei. Bei \texttt{.py}-Dateien zeigt Visual Studio Code flake8-Warnungen als gelbe Markierungen an, die Sie durch einen Mouseover erklärt bekommen.

    \item Das Format der Abgabe muss die folgenden Eigenschaften haben:
    %
    \begin{itemize}
        \item Es handelt sich um eine Archivdatei (.zip, .gz, .tar.gz), in der alle Dateien zu den einzelnen Übungsaufgaben sowie die Datei \texttt{mitglieder.txt} enthalten ist. 
        \item Das Verzeichnis hat den Namen
        \lstinline{<NummerZettel>_<Nachname1>_<Nachname2>_<Nachname3>.<FileExtension>} mit den jeweiligen Nachnamen ihrer Gruppenmitglieder, beispielsweise also \lstinline{01_Reuter_Gertz_Schmidt.zip}.
    \end{itemize}
\end{enumerate}
